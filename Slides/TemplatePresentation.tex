% ### Uses XeLaTeX ### %
% ### Needs beamer-master ### %
\documentclass[aspectratio=169]{beamer} %. Aspect Ratio 16:9
\usetheme{AI2} % beamerthemeSprace.sty
% DATA FOR FOOTER
\date{2019}
\title{}
\author{}
\institute{Advanced Institute for Artificial Intelligence (AI2)}
\begin{document}    
% ####################################
% FIRST SLIDE 						:: \SliTit{<Title of the Talk>}{<Author Name>}{<Intitution>}
% SLIDE SUB-TITLE					:: \SliSubTit{<Title of the Chapter>}{<Title of the Section>}
% SLIDE WITH TITLE 					:: \SliT{<Title>}{Content}
% SLIDE NO TITLE 						:: \Sli{<Content>} 
% SLIDE DOUBLE COLUMN WITH TITLE 	:: \SliDT{<Title>}{<First Column>}{<Second Column>}
% SLIDE DOUBLE COLUMN NO TITLE 		:: \SliD{<First Column>}{<Second Column>}
% SLIDE ADVANCED WITH TITLE 			:: \SliAdvT{<Title>}{<Content>}
% SLIDE ADVANCED  NO TITLE 			:: \SliAdv{<Content>}
% SLIDE ADVANCED DOUBLE TITLE 		:: SliAdvDT{<Title>}{<First Column>}{<Second Column>}
% SLIDE ADVANCED DOUBLE NO TITLE 	:: SliAdvD{<First Column>}{<Second Column>}
% ITEMIZE 							:: \begin{itemize}  \IteOne{1st Level} \IteTwo {2nd Level} \IteThr{3rd Level} \end{itemize}
% SECTION 							:: \secx{Section} | \secxx{Sub-Section}
% COLOR BOX 						:: \blu{blue} + \red{red} + \yel{yellow} + \gre{green}
% FRAME 							:: \fra{sprace} \frab{blue} \frar{red} + \fray{yellow} + \frag{green}	
% REFERENCE						:: \refer{<doi number>}
% FIGURE 							::  \img{X}{Y}{<scale>}{Figures/.png} 
% FIGURE							:: \begin{center}\includegraphics[scale=<#>]{Figures/.png}\end{center}
% PROJECT STATUS					:: \planned\~    \started\~   \underway\~   \done\~   
% EXERCICIO							:: \Exe{<#>}{<text>}
% STACKREL							:: \underset{<down>}{<up>}
% FLUSH LEFT						:: \begin{flalign*}  & <1st equation> & \\  & <12nd equation>  & \\ \end{flalign*}
% REAL / IMAGINAY					:: \Re / \Im
% SLASH								:: \sl{} or \sl
% BOLD MATH							:: \pmb{<>}
% ####################################
%
% FIRST SLIDE :: DO NOT BREAK LINE !!!
\SliTit{Python}{Advanced Institute for Artificial Intelligence}{https://advancedinstitute.ai}

% SLIDE WITH TITLE
\SliT{Sumario}{

\begin{itemize}
  \IteOne{Introdução}
  \IteOne{Estruturas e Função de Controle}
  \IteOne{Coleções}
  \IteOne{Programação Orientada a Objetos}
  \IteOne{Manipulação de arquivos}
  \IteOne{Processos e Threading}  
\end{itemize}

}

% SLIDE WITH TITLE
\SliT{Introdução}{

\begin{itemize}
  \IteOne{Python é uma linguagem interpretada}
  \IteOne{Caminho do Python no Sistema}
  \IteTwo{which python}
  \IteOne{Versão do Python}
  \IteTwo{python -V}

\end{itemize}

}

% SLIDE WITH TITLE
\SliT{Usando Python}{

\begin{itemize}
  \IteOne{Iniciando interpretador Python}
  \IteTwo{python}
  \IteTwo{Python 3.6.8 |Anaconda, Inc.| (default, Dec 30 2018, 01:22:34) }
  \IteTwo{[GCC 7.3.0] on linux}
  \IteTwo{Type "help", "copyright", "credits" or "license" for more information.}
  \IteTwo{$>>>$ Esse é o prompt para receber comandos python }
  \IteOne{Ctrl+D sai do interpretador }
\end{itemize}

}

% SLIDE WITH TITLE
\SliT{Usando Python}{

Comando print

\begin{itemize}
  \IteOne{print "hello world"}
  \IteTwo{Em python 2 é possível utilizar dessa forma:}
  \IteOne{print "hello world"}
  \IteTwo{Em python 3 é obrigatório utilizar $($ $)$}
  \IteOne{print ("hello world")}
\end{itemize}

Comentários no código
\begin{itemize}
  \IteOne{# : comentando uma linha}
  \IteOne{''' : começar e terminar bloco de comentário}
  \IteOne{""" : começar e terminar bloco de comentário}
\end{itemize}

}

% SLIDE WITH TITLE
\SliT{Usando Python}{

Indentação

\begin{itemize}
  \IteOne{O controle de início e fim deblocos de código é feito por meio de Indentação}
  \IteOne{Indentação pode ser controlada por um tamanho fixo de espaços em branco}
  \IteOne{Exemplo}
  \IteTwo{print ("teste")}
  \IteTwo{if (i == 0):}
  \IteTwo{\hspace{1cm} print ("0")}
  \IteTwo{else:}
  \IteTwo{\hspace{1cm} print ("outro valor")}
  \IteTwo{\hspace{1cm} if (i $>$= 0):}
  \IteTwo{\hspace{2cm} print ($>$=0")}

\end{itemize}


}
% SLIDE WITH TITLE
\SliT{Tipos de dados - Números}{

Existem três tipos numéricos em python: números inteiros, números de ponto flutuante e números complexos. 

\begin{itemize}
  \IteOne{Booleanos são um subtipo de números inteiros. }
  \IteOne{Inteiros têm precisão ilimitada.}
  \IteOne{Números de ponto flutuante são geralmente implementados usando tipo Double em C}
\end{itemize}

}

\SliT{Tipos de dados - Strings}{

Strings podem ser manipuladas de diversas maneiras em Python

\begin{itemize}
  \IteOne{podem ser representadas usando aspas simples ' ' ou aspas duplas " "  }
  \IteOne{É possível utilizar catacteres escape   }

\end{itemize}

}


\SliT{Funções}{

\begin{itemize}
  \IteOne{A palavra-chave def é usada para definir funções}
  \IteOne{Deve ser definida antes de ser utilizada}
  \IteOne{O valor de retorno padrão é None}
\end{itemize}

}

\SliT{Função}{
Argumento pode ser gerado da seguinte forma:
\begin{itemize}
  \IteOne{nome de variável}
  \IteOne{nome de variável e tipo padrão}
\end{itemize}

Escopo de variável
\begin{itemize}
  \IteOne{variáveis possuem escopo local ao bloco onde são criadas}
  \IteOne{Pode ser definidas variáveis globais}
\end{itemize}

}

\SliT{Funções}{

Função sem argumentos:
\hfill \break
\hfill \break
def greeting():

\hspace{1cm} print("hello world")

\hfill \break
\hfill \break
greeting()

}

\SliT{Argumento de Função}{

def numsquare(num):

\hspace{1cm} return num * num

\hfill \break
number=10
\hfill \break
numsquare(number)

\hfill \break
def numsquare(num=10):

\hspace{1cm} return num * num

\hfill \break
numsquare()

}

\SliT{Obtendo dados do usuário}{

função input() é utilizada para aguardar um valor digitado no terminal pelo usuário.
\hfill \break
\hfill \break
usrip = input("número inteiro: ")
\hfill \break
usrnum = int(usrip)
\hfill \break
sqrnum = numsquare(usrnum)
\hfill \break
print("Square of entered number is: {}".format(sqrnum))
\hfill \break
\hfill \break
usrip = input("float: ")
\hfill \break
usrnum = float(usrip)
\hfill \break
sqrnum = numsquare(usrnum)
\hfill \break
print("Square of entered number is: {}".format(sqrnum))
\hfill \break
\hfill \break
usrname = input("nome: ")
\hfill \break
print("nome: ",usrname)

}

\SliT{Usando bibliotecas adicionais}{

A palavra reservada import permite adicionar pacotes que não são nativos do Python
\hfill \break
\hfill \break
import subprocess
\hfill \break
\# Executa um comando linux no terminal
\hfill \break
subprocess.call('date')
\hfill \break
\hfill \break
A palavra reservada from premite importar apenas parte de um pacote 
\hfill \break
\hfill \break
exemplo:
\hfill \break
\hfill \break
from sklearn.model\_selection import train\_test\_split
}

\SliT{Estruturas e Função de Controle}{

Argumento pode ser gerado da seguinte forma:
\begin{itemize}
  \IteOne{if}
  \IteOne{for}
  \IteOne{while}
\end{itemize}

}

\SliT{Estruturas e Função de Controle}{

if 
\begin{itemize}
  \IteOne{As instruções if avaliam uma condição, caso seja verdadeira executa o bloco seguinte}
  \IteOne{Pode ser combinado com uma estrutura else, que é executada quando a condição não é verdadeira no bloco if}
\end{itemize}

Exemplo:

var = 100
\hfill \break
if (var==100):

\hspace{1cm} print("100")
\hfill \break
else: 
\hspace{1cm} print("not 100")

}

\SliT{Estruturas e Função de Controle}{

for
\begin{itemize}
    \IteOne{executam um certo bloco de código para um número conhecido de iterações. }
    \IteOne{Um bloco de código pode ser executado para o número de itens existentes em uma lista, dicionário, variável de sequência ou tupla}
    \IteOne{Um bloco de código pode ser executado em um intervalo contado de etapas}
\end{itemize}

Exemplo:
\hfill \break

a=(10,20,30,40,50)
\hfill \break
for b in a:
\break
\hspace{1cm} print "square of " + str(b) + " is " +str(b*b)

}


\SliT{Estruturas e Função de Controle}{

while

\begin{itemize}
    \IteOne{O loop while é executado enquanto uma declaração condicional retorna true }
    \IteOne{A instrução condicional é avaliada toda vez que um bloco de código é executado }
    \IteOne{A execução para no momento em que a instrução condicional retorna false. }
\end{itemize}

Exemplo:
\hfill \break

count = 0
\hfill \break
while (count $<$ 9):
    \hfill \break
    \hfill \break
    \hspace{1cm} print("iteração",count)
    \hfill \break
    \hspace{1cm} count+=1


}










\end{document}